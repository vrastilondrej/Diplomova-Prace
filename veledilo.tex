\documentclass[FP,DP]{tulthesis}
\hypersetup{
    hidelinks = true,
}
\usepackage{blindtext}
\usepackage{url}
\usepackage[obeyFinal]{easy-todo}
\usepackage{polyglossia}
\setdefaultlanguage{czech}

% bibliografie
\usepackage{natbib}
\bibliographystyle{agsm}


% fonty
\usepackage{fontspec}
\usepackage{xunicode}
\usepackage{xltxtra}
%\setmainfont[Mapping=tex-text,BoldFont={* Bold},Numbers=OldStyle]{Baskerville 10 Pro}
%\setsansfont[Mapping=tex-text,BoldFont={* Bold},Numbers=OldStyle]{Myriad Pro}
%\setmonofont[Scale=MatchLowercase]{Vida Mono 32 Pro}

% příkazy specifické pro tento dokument
\newcommand{\argument}[1]{{\ttfamily\color{\tulcolor}#1}}
\newcommand{\prikaz}[1]{\argument{\textbackslash #1}}
\newenvironment{myquote}{\begin{list}{}{\setlength\leftmargin\parindent}\item[]}{\end{list}}
\newenvironment{listing}{\begin{myquote}\color{\tulcolor}}{\end{myquote}}
\sloppy

% deklarace pro titulní stránku
\TULtitle{Programování v pregraduální přípravě učitelů informatiky}{Programming in undergraduate training of CS teachers}
\TULprogramme{N1101}{Matematika}{Mathematics}
\TULbranch{7504T077}{Učitelství informatiky pro střední školy}{Teacher training for lower and upper-secondary school, Informatics}
\TULbranch{7504T089}{Učitelství matematiky pro střední školy}{Teacher Training for Upper Secondary Schools, Mathematics}
\TULauthor{Bc. Ondřej Vraštil}
\TULsupervisor{Mgr. Jan Berki, Ph.D.}
\TULyear{2016}


% příkazy
\newcommand{\verze}{1.3}
\newcommand{\chapquote}[3]{\begin{quotation} \textit{#1} \end{quotation} \begin{flushright} - #2, \textit{#3}\end{flushright} }





\begin{document}

\ThesisStart{male}

\begin{abstractCZ}
Tato zpráva popisuje třídu \texttt{tulthesis} pro sazbu absolventských prací
Technické univerzity v~Liberci pomocí typografického systému \LaTeX.
\end{abstractCZ}

\vspace{2cm}

\begin{abstractEN}
This report describes the \texttt{tulthesis} package for Technical university of
Liberec thesis typesetting using the \LaTeX\ typographic system.
\end{abstractEN}

\clearpage

\begin{acknowledgement}
Rád bych poděkoval všem, kteří přispěli ke vzniku tohoto dílka.
\end{acknowledgement}

\tableofcontents

\clearpage

%\begin{abbrList}
%\textbf{TUL} & Technická univerzita v~Liberci \\
%\end{abbrList}
\chapter{Úvod}


Nedilnou součastí pripravy budoucich ucitelu informatiky na základních a střednich školách  je vyuka programovani a algoritmizace. Toto téma je nejen obsaženo v rámcových vzdělávacíh programů pro gymnazia, ale stává se i moderním trendem vyučovaní informatiky na základních školách. Ucitel můze znalosti vyuzit nejen pri vyuce samotného programovani, ale i v pridruzenych oblastech jako v robotice nebo v pocitacove vede, ktera se v ruznych formach zacina na nasich zakladnich a strednich skolach objevovat. Nutnou podminkou spravně provedené didaktické transformace látky směrem  k zakovi je nadhled a vzdelani ucitele. Jednou z hlavnich premis je kvalitni vzdělani v ramci pregradualní pripravy, ve ktere by měl student – budouci učitel získat uvod do tématu a vhled do tématu natolik, aby dokazal obstojně predat znalosti svym zakum. Na kvalitu pregraduální pripravy ma vliv mnoho aspektu, jednim z nich je i relevance a aktualita, které mají pro informatiku, jakožto mladý a dynamicky se rozvíjející obor velký význam. Fakulty připravující učitele informatiky by měli pružně reagovat na moderni trendy ve výuce a uspokojit poptavku po kvalifikovaných učitelích.  V dnesni době se na našich zakladnich a strednich skolach zacinaji vyucivat pro školni prostredi nova temata jako je robotika nebo unplugged teaching, pro která je znalost algoritmizace nutná. Jak si naše vysoké školy vedou ve výuce tohoto oboru? Následují moderní trendy a požadavky zaměstnavatelů budoucích absolventů? Je pořadí předmětů během studia smysluplné? Dá se vysledovat podobnost mezi programy napříč republikou? Existuje  jeden nejvhodnější způsob jak učit programovaní budoucí učitele, nebo je možnost volit z více cest? Abychom tyto otázky mohli zodpovědět, je v první řadě nutné pokusit se analyzovat zdroje týkající se programování a pregraduální přípravy učitelů. Dále je potřeba získat data o obsahu jednotlivých předmětů v rámci pregraduální přípravy napříč fakultami v ČR. Fakulty tyto data zveřejňují v sylabech, které jsou volně k dispozici na stránkách jednotlivých fakult. Získaná data budou zkoumána fornou textové analýzy, jež by mohla na výše zmíněné otázky odpovědět. Na výsledcích teoreticko-rešeršní části i praktické části bude postavena závěrečná část, návrh idealního sestavení vhodného konceptu výuky programování na VŠ pro budoucí učitele středních i základních škol,,slovo{}``

% KAPITOLA PROGRAMOVÁNÍ
% **********************
\listoftodos
\chapter{Programování}
Pokud máme zkoumat pregradualní přípravu učitelů informatiky v oblasti programování, měli bychom odpovědět na důležitou otázku -- Proč se vlastně zařazuje výuka programování do přípravy budoucích učitelů informatiky? Jaký zde má smysl? Odpověd na tuto otázku není jednoduchá, proto si ji budeme strukturovat -- podíváme se na učitele informatiky a jeho znalost programování z několika pohledů, které ho do různé míry definují. 

\todo{Programování z pohledu naší společnosti.}\todo{Programování z  pohledu RVP}
\section{Programování z pohledu RVP}
Jedním z určujících dokumentů pro vzdělávání na základních a středních školách je Rámcový vzdělávací program. Tento koncepční dokument určuje a specifikuje obsah výuky na republikové úrovni a odvyjí se od něho dokumenty na úrovni nižší -- školní vzdělácací programy, má tedy zásadní vliv. V jaké míře je zde programování nebo algoritmizace obsažena?
 
V RVP pro základní vzdělávání \citep{rvpzv} spadá pod tzv. vzdělávací oblast Informační a komunikační \textit{technologie}. Charakteristika oblasti pojem algoritmus, programování ani jím příbuzná nebo odvozená slova neuvádí. Zmínku najdeme tzv. cílových zaměřeních kdy je uvedeno, že \textit{Vzdělání v oblasti vede žáka k schopnosti formulovat svůj požadavek a využívat při interakci s počítačem algoritmické myšlení}. RVP dále kategorizuje vzdělávací obsah a rozděluje ho do učiva prvního a druhého stupně, pojem algoritmizace ani programování zde nenajdeme. 

V RVP progymnázia\citep{rvpgy}




\section{Má smysl učit programování?}
\section{Jaké jsou koncepty výuky programování?}
\section{Jak se u nás programování učí?}
\chapter{Kurikulární výzkum}
\section{Stav problematiky}
\chapter{Textová analýza}
\section{Popis vzorku}
\section{Popis způsobu analýzy}
\section{Výsledky}
\chapter{Návrh konceptu pregraduální přípravy}
\section{ZŠ}
\section{SŠ}
\chapter{Závěr}
\textcolor{gray}{\Blindtext}


\bibliography{citace}


%\begin{quote}
%l,,Programming a computer means nothing more or less than communicating to it in a language
%that it and the human user can both understand. ``\citep{Češka1994}
%\end{quote}
%\chapquote{,,Programming a computer means nothing more or less than \mbox{communicating} to it in a language
%that it and the human user can both understand. ``}{Lewis Carroll}{Alice in Wonderland}
\end{document}
