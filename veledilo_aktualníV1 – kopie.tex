\documentclass[FP,DP]{tulthesis}
%\hypersetup{
  %  hidelinks = true,
%}
%\usepackage{hyperref}
\usepackage{blindtext}
\usepackage{url}
\usepackage[obeyFinal]{easy-todo}
\usepackage{polyglossia}
\usepackage{enumitem}
\usepackage{array}
\newcolumntype{L}[1]{>{\raggedright\let\newline\\\arraybackslash\hspace{0pt}}m{#1}}
\usepackage{tikz}
\usepackage{graphicx}
\usepackage{svg}
\usepackage{fancyvrb}
\usepackage{tabularx}

\def\UrlBreaks{\do\/\do-}

\setdefaultlanguage{czech}

% bibliografie
\newcommand*{\biber}{\sty{biber}\xspace}
\newcommand*{\biblatex}{\sty{biblatex}\xspace}

\usepackage[
   backend=biber
  ,bibstyle=iso-authoryear
  ,citestyle=authoryear
  ,autolang=other
  ,pagetotal=true
  ,sortlocale=cs_CZ
  ,bibencoding=UTF8
  ,spacecolon=false
  %,block=ragged
]{biblatex}

\addbibresource{mybib.bib}
%\setcitestyle{round,aysep={},yysep={;},aysep={,}}



% fonty
\usepackage{fontspec}
\usepackage{xunicode}
\usepackage{xltxtra}
%\setmainfont[Mapping=tex-text,BoldFont={* Bold},Numbers=OldStyle]{Baskerville 10 Pro}
%\setsansfont[Mapping=tex-text,BoldFont={* Bold},Numbers=OldStyle]{Myriad Pro}
%\setmonofont[Scale=MatchLowercase]{Vida Mono 32 Pro} ah

%TIKZ
\usetikzlibrary{calc,trees,positioning,arrows,chains,shapes.geometric,%
    decorations.pathreplacing,decorations.pathmorphing,shapes,%
    matrix,shapes.symbols}
\tikzstyle{startstop} = [rectangle, rounded corners, minimum width=1cm, minimum height=1cm,text centered, draw=black,text width=2cm,fill=orange!20]
\tikzstyle{io} = [rectangle, rounded corners, minimum width=1cm, minimum height=1cm,text centered, draw=black,text width=2cm,fill=orange!20]
\tikzstyle{process} = [rectangle, minimum width=1cm, minimum height=0.8cm, text centered, draw=black, fill=orange!30,text width=2.5cm]
\tikzstyle{decision} = [rectangle, rounded corners, minimum width=1cm, minimum height=0.8cm,text centered, draw=black,text width=2cm]
\tikzstyle{arrow} = [thick,->,>=stealth]
\tikzstyle{arrow2} = [thick,-,>=stealth]

% příkazy specifické pro tento dokument
\newcommand{\argument}[1]{{\ttfamily\color{\tulcolor}#1}}
\newcommand{\prikaz}[1]{\argument{\textbackslash #1}}
\newenvironment{myquote}{\begin{list}{}{\setlength\leftmargin\parindent}\item[]}{\end{list}}
\newenvironment{listing}{\begin{myquote}\color{\tulcolor}}{\end{myquote}}
\sloppy

% deklarace pro titulní stránku
\TULtitle{Programování v pregraduální přípravě učitelů informatiky[verze 16. 5.] }{Programming in undergraduate training of CS teachers}
\TULprogramme{N1101}{Matematika}{Mathematics}
\TULbranch{7504T077}{Učitelství informatiky pro střední školy}{Teacher training for lower and upper-secondary school, Informatics}
\TULbranch{7504T089}{Učitelství matematiky pro střední školy}{Teacher Training for Upper Secondary Schools, Mathematics}
\TULauthor{Bc. Ondřej Vraštil}
\TULsupervisor{Mgr. Jan Berki, Ph.D.}
\TULyear{2016}


% příkazy
\newcommand{\verze}{1.3}
\newcommand{\chapquote}[3]{\begin{quotation} \textit{#1} \end{quotation} \begin{flushright} - #2, \textit{#3}\end{flushright} }

\newcommand{\ahoj}[2]{\begin{quotation} \textit{#1} \end{quotation} \begin{flushright} - \textit{#2}\end{flushright} }



\usepackage{booktabs}

\begin{document}

\ThesisStart{male}

\begin{abstractCZ}
Tato zpráva popisuje třídu \texttt{tulthesis} pro sazbu absolventských prací
Technické univerzity v~Liberci pomocí typografického systému \LaTeX.
\end{abstractCZ}

\vspace{2cm}

\begin{abstractEN}
This report describes the \texttt{tulthesis} package for Technical university of
Liberec thesis typesetting using the \LaTeX\ typographic system.
\end{abstractEN}

\clearpage

\begin{acknowledgement}
Rád bych poděkoval všem, kteří přispěli ke vzniku tohoto dílka.
\end{acknowledgement}

\tableofcontents

\clearpage

%\begin{abbrList}
%\textbf{TUL} & Technická univerzita v~Liberci \\
%\end{abbrList}
\chapter{Úvod}
Nedílnou součástí pregraduální přípravy budoucích učitelů informatiky základních i~středních škol je výuka programováni a teorie algoritmů. Toto téma je v~součastnosti získává na popularitě i na středních a základních školách, kde je mu věnováno stále více pozornosti. Výuka úvodu do programování a algoritmického myšení je nejen obsažena v rámcovém vzdělávacím programu pro gymnázia a některé odborné střední školy, ale v uzpůsobené podobě se objevuje~i ve výuce na školách základních \parencite{t03}, kde se mohou mladí žáci setkat se speciálními programovacími jazyky pro děti.


\begin{table}[ht]
\hyphenpenalty=10000
\scriptsize
\center
    \begin{tabular}{L{2cm} L{4cm} L{1cm} L{5cm}}
   \specialrule{.15em}{.05em}{.05em}  \textbf{univerzita}              & \textbf{fakulta}    &\textbf{typ }                            & \textbf{program}                   \\ \specialrule{.15em}{.05em}{.05em} 
UJEP&  Přírodovědecká fakulta & Bc.                   & Informatika            \\ \hline
ZČU & Fakulta pedagogická& Bc.& Informatika se zaměřením na vzdělávání\\ 
&Fakulta pedagogická& NMgr.&Učitelství informatiky pro ZŠ\\ \hline
JU & Pedagogická fakulta&Bc.& Informační technologie se zaměřením na vzdělávání \\
JU & Pedagogická fakulta&NMgr.&Učitelství informatiky pro 2. stupeň základních škol \\
JU & Přírodovědecká fakulta&Bc.&Informatika pro vzdělávání \\ 
JU & Přírodovědecká fakulta&NMgr.&Učitelství informatiky pro střední školy\\ \hline
MUNI & Fakulta informatiky&Bc.&Informatika a druhý obor \\ 
MUNI & Fakulta informatiky&NMgr.&Učitelství informatiky pro střední školy\\ \hline
UPOL & Přírodovědecká fakulta&Bc.&Informatika pro vzdělávání\\
UPOL & Přírodovědecká fakulta&NMgr.&Učitelství informatiky pro střední školy\\ \hline
OU & Přírodovědecká fakulta&Bc.&Informatika (dvouoborové)\\
OU & Přírodovědecká fakulta&NMgr.&Učitelství informatiky pro 2. stupeň základních škol a střední školy\\
OU & Pedagogická fakulta&Bc.&Informační a komunikační technologie se zaměřením na vzdělávání\\ \hline
UHK &Přirodovědecká fakulta&Bc.&Informatika se zaměřením na vzdělávání\\
UHK & Pedagogická fakulta&NMgr.&Učitelství pro 2. stupeň ZŠ - informatika\\
UHK &Pedagogická fakulta&NMgr.&(Učitelství pro střední školy - informatika)\\ \hline
TUL & Přírodovědně-humanitní a pedagogická&Bc.&Informatika se zaměřením na vzdělávání\\
TUL& Přírodovědně-humanitní a pedagogická&NMgr.&Učitelství informatiky pro střední školy\\
TUL& Přírodovědně-humanitní a pedagogická&NMgr.&Učitelství informatiky pro 2. stupeň základní školy\\ \hline
UK&Matematicko-fyzikální fakulta&Bc.&Informatika se zaměřením na vzdělávání\\
UK&Matematicko-fyzikální fakulta&NMgr.&Učitelství informatiky\\
UK&Pedagogická fakulta&Bc.&Informační technologie\\
UK & Pedagogická fakulta&NMgr.&Informační a komunikační technologie\\ \hline
UTB&Fakulta aplikované informatiky&NMgr.&Učitelství informatiky pro střední školy\\
\specialrule{.15em}{.05em}{.05em} 
    \end{tabular}
\end{table}






\chapter{Textová analýza}
\section{Popis vzorku}
\section{Popis způsobu analýzy}
\section{Výsledky}
\chapter{Návrh konceptu pregraduální přípravy}
\section{ZŠ}
\section{SŠ}
\chapter{Závěr}
\textcolor{gray}{\Blindtext}
\printbibliography[title={Reference bibliography},heading={bibnumbered}]

\listoftables

%\begin{quote}
%l,,Programming a computer means nothing more or less than communicating to it in a language
%that it and the human user can both understand. ``\citep{Češka1994}
%\end{quote}
%\chapquote{,,Programming a computer means nothing more or less than \mbox{communicating} to it in a language
%that it and the human user can both understand. ``}{Lewis Carroll}{Alice in Wonderland}
\end{document}
