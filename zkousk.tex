\documentclass{minimal}
\usepackage{tikz}
\usepackage{verbatim}
\usepackage{polyglossia}
\setdefaultlanguage{czech}
\hyphenpenalty=10000
\hbadness=10000


\begin{comment}
:Title: Assignment structure

The structure of an assignment illustrated using a flow chart. Labels are in Danish.

\end{comment}
\usetikzlibrary{calc,trees,positioning,arrows,chains,shapes.geometric,%
    decorations.pathreplacing,decorations.pathmorphing,shapes,%
    matrix,shapes.symbols}
\tikzstyle{startstop} = [rectangle, rounded corners, minimum width=1cm, minimum height=1cm,text centered, draw=black,text width=2cm,fill=orange!20]
\tikzstyle{io} = [rectangle, rounded corners, minimum width=1cm, minimum height=1cm,text centered, draw=black,text width=2cm,fill=orange!20]
\tikzstyle{process} = [rectangle, minimum width=1cm, minimum height=0.8cm, text centered, draw=black, fill=orange!30,text width=2.5cm]
\tikzstyle{decision} = [rectangle, rounded corners, minimum width=1cm, minimum height=0.8cm,text centered, draw=black,text width=2cm]
\tikzstyle{arrow} = [thick,->,>=stealth]
\tikzstyle{arrow2} = [thick,-,>=stealth]



\begin{document}
\tikzset{every picture/.append style={font=\fontsize{9}{12}\selectfont}}

\begin{tikzpicture}[node distance=2cm]
 \node (start) [startstop] {analýza problému};
  \node (m2) [io,right of=start,xshift=0.8cm] {jazyk na zápis řešení};
  \node (m3) [io,right of=m2,xshift=0.8cm] {řešení problému pomocí:};
\node (v1) [decision,below of=m3,xshift=3cm,yshift=-1cm] {proměných};
\node (v2) [decision,below of=m3,yshift=-1cm] {větvení};
\node (v3) [decision,below of=m3,xshift=-3cm,yshift=-1cm] {cyklů};
\node (v4) [io,below of=m3,xshift=-6cm,yshift=-1cm] {posloupnosti příkazů};
\node (v5) [decision,below of=m3,xshift=6cm,yshift=-1cm] {interakčních nástrojů};
  \node (m4) [io,right of=m3,xshift=0.8cm] {hledání a~opravování chyb};
  \node (m5) [io,right of=m4,xshift=0.8cm] {interpretace a~zápis řešení};
\draw [arrow] (start) -- (m2);
\draw [arrow] (m2) -- (m3);
\draw [arrow] (m3) -- (m4);
\draw [arrow] (m4) -- (m5);
\draw [arrow2] (m3) -- (v1);
\draw [arrow2] (m3) -- (v2);
\draw [arrow2] (m3) -- (v3);
\draw [arrow2] (m3) -- (v4);
\draw [arrow2] (m3) -- (v5);
  \end{tikzpicture}
\end{document}
%%% Local Variables: 
%%% mode: latex
%%% TeX-master: t
%%% End: